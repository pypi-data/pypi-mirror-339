
\documentclass[titlepage, oneside, a4paper]{article}

\usepackage {import}
\usepackage[margin=25mm, left=17mm, right=17mm]{geometry}
\usepackage[utf8]{inputenc}
\usepackage[english]{babel}
\usepackage{amsmath}
\usepackage{amssymb}
\usepackage{fancyhdr}
\usepackage{comment}
\usepackage{olymp}
\usepackage{epigraph}
\usepackage{expdlist}
\usepackage{graphicx}
\usepackage{ulem}
\usepackage{lastpage}

%- for problem in problems
%- if problem.blocks.preamble is defined
\VAR{problem.blocks.preamble}
%- endif
%- endfor

\def\showSourceFileName{1}
\newif\ifintentionallyblankpages % comment this out to add blank pages

\pagestyle{fancy}
\fancyhf{}
\renewcommand{\footrulewidth}{0.4pt}

%- if contest is defined
\title{\VAR{contest.title or ""}}
\author{\VAR{contest.location if contest.location is defined else ""}}
\date{\VAR{contest.date if contest.date is defined else ""}}
%- else
\title{}
\author{}
\date{}
%- endif

\makeatletter
\let\newtitle\@title
\let\newauthor\@author
\let\newdate\@date
\makeatother

\rhead{\newtitle}
\lhead{\newauthor}
\lfoot{\newdate}
\rfoot{\thepage}

\begin{document}

%- if contest is defined
\begin{frontpage}{\newtitle}{\newauthor}{\newdate}
	\begin{itemize}
		\item Sobre a prova:
		      \begin{enumerate}
			      \item A prova tem duração de 5 horas;
		      \end{enumerate}
		\item Sobre a entrada:
		      \begin{enumerate}
			      \item A entrada do seu programa deve ser lida da entrada padrão (standard input);
			      \item Quando uma linha da entrada contém vários valores, estes são separados por um único espaço em branco, a menos que explicitado o contrário no enunciado do problema;
			      \item Toda linha da entrada terminará com um caractere final-de-linha.
		      \end{enumerate}

		\item Sobre a saída:
		      \begin{enumerate}
			      \item A saída de seu programa deve ser escrita na saída padrão;
			      \item Quando uma linha da saída contém vários valores, estes devem ser separados por um único espaço em branco, a menos que explicitado o contrário no enunciado do problema;
			      \item Toda linha da saída deve terminar com um caractere final-de-linha.
		      \end{enumerate}

		\item Sobre as submissões:
		      \begin{enumerate}
			      \item Você deve submeter o código fonte das suas soluções;
			      \item Os comandos utilizados para compilação serão:
			            \begin{enumerate}
				            %- for language in languages:
				            \item \VAR{language.name | escape}: \VAR{language.command | escape}
				                  %- endfor
			            \end{enumerate}
			      \item Para linguagem Java, o nome da classe principal do arquivo deve ser \textbf{Main}. A classe deve ser pública.
		      \end{enumerate}
		      \centering\vspace*{\fill}\textbf{Este caderno contém \VAR{problems | length} problema(s) e \pageref{LastPage} página(s)} \\
		      \vspace*{\fill}
	\end{itemize}
\end{frontpage}
%- endif

%- for problem in problems
\subimport{\VAR{problem.path | parent}/}{\VAR{problem.path | stem}}
%- endfor

\end{document}
